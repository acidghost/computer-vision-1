\documentclass[11pt]{article}
\usepackage[utf8]{inputenc}
\usepackage{graphicx}
\usepackage{url}
\usepackage{amsmath}
\usepackage{float}


\title{
	{Computer Vision 1 - Assignment 2 \\
	Linear Filters: Gaussians and Derivatives}
}
\author{
Selene Baez Santamaria (10985417) - Andrea Jemmett (11162929)}
\date{\today}

\begin{document}

\maketitle


\section{1D Gaussian Filter}
% Question: "Compare your function with MATLAB’s built-in fspecial. What are the
% differences?"
Our function takes the same arguments as the built-in fspecial, but it generates
different output. While our function generates a vector of size $kernelLength$,
the \emph{fspecial} function returns a rectangular or squared matrix dependant on the
$hsize$ parameter.


\section{Convolving an image with a 2D Gaussian}
% Question 1: Show that the outputs of convolution with the 2D kernel (i.e. the
% one corresponding to the two 1D kernels) and convolution with 1D kernels are
% essentially the same.  Use kernel length of 11.  Report the numerical
% equivalence as well as plots at every stage of filtering pipeline.  Question
% 2: Explain the benefits of kernel separability, shortly.
We use a value of $\sigma_x = 2$ and $\sigma_y = 2$ and a kernel length of 11
(as requested). First we call the \emph{gaussian} function defined in the
previous section to get two 1D filters, one with each $\sigma$. We multiply
those 1D vectors to obtain a 2D gaussian kernel.

To convolve the images with kernels we use the built-it function $conv2$. After
testing the different options we found out the following:

\begin{description}
		\item[full] performs the full convolution. Visually, we note that the
						produced images have black frames on the edges since the image size
						is slightly (10 pixels) larger than the original image; this is due
						to the fact that the convolution operation ``spreads'' pixel
						intensities over neighboring pixels overflooding outside the
						original image size;
		\item[same] returns the central part of the convolution, that is the image
						returned as output is of the same size of the input matrix $A$;
		\item[valid] performs convolution but omits those parts of the convolution
						that are computed not exclusively using image pixels (image edges);
						the resulting image is indeed smaller than the original.
\end{description}

In order to check that both functions produce the same images we make a pixel
per pixel numeric comparison. We check that the absolute value of the difference among
the values is less that a small value $\epsilon = e^{-12}$. Hereby we show the
original and filtered images using the \emph{same} option:

\begin{figure}[H] \centering
	\includegraphics[width=1\textwidth]{imgs/flowers_conv.jpg}
	\caption{Comparison for flowers.jpg. From left to right: original image, image
	filtered with 2D kernel and image filtered with 1D kernel. All images have a
	$\sigma$ value of 2}
	\label{fig:flowers}
\end{figure}

\begin{figure}[H] \centering
	\includegraphics[width=.9\textwidth]{imgs/flowers_col_heatmap.jpg}
	\caption{Heatmap showing the absolute difference per pixel between Step
		1 of 1D filtering and 2D final filtered image for flowers.jpg. Reported
		mean error per pixel is $0.0115$}
	\label{fig:flowers_col_heatmap}
\end{figure}

\begin{figure}[H] \centering
	\includegraphics[width=.9\textwidth]{imgs/flowers_heatmap.jpg}
	\caption{Heatmap showing the absolute difference per pixel between both
		final filtered images for flowers.jpg. Reported mean error per pixel is
		$4.0687 e^{-17}$}
	\label{fig:flowers_heatmap}
\end{figure}

\begin{figure}[H] \centering
	\includegraphics[width=1\textwidth]{imgs/zebra_conv.jpg}
	\caption{Comparison for zebra.png (original, 2D kernel, 1D kernel with
					$\sigma=2$). Reported mean error per pixel is $5.5609 e^{-17}$}
	\label{fig:zebra}
\end{figure}

\begin{figure}[H] \centering
	\includegraphics[width=.9\textwidth]{imgs/zebra_col_heatmap.jpg}
	\caption{Heatmap showing the absolute difference per pixel between Step
		1 of 1D filtering and 2D final filtered image for zebra.png. Reported
		mean error per pixel is $0.0223$}
	\label{fig:zebra_col_heatmap}
\end{figure}

\begin{figure}[H] \centering
	\includegraphics[width=.9\textwidth]{imgs/zebra_heatmap.jpg}
	\caption{Heatmap showing the absolute difference per pixel between both
		final filtered images for zebra.png. Reported mean error per pixel is
		$5.5609 e^-017$}
	\label{fig:zebra_heatmap}
\end{figure}

% We note that the 1D convoluted image takes longer to compute.
Having kernel separability is helpful since the number of operations needed to
perform the convolution operation are fewer than with a full 2D kernel. For
example filtering an M-by-N image with a P-by-Q kernel would require $MNPQ$
operations. If the kernel is separable, we can apply the convolution in an
ordered fashion, first by applying a vector kernel to columns and then filtering
the results using a vector kernel on the rows. Filtering in these two steps
requires respectively $MNP$ and $MNQ$ operations, so the total number of
operations is $MN(P + Q)$, which is less than $MNPQ$.

\section{Gaussian Derivative}


The Gaussian derivative may be used to find edges in am image. An application
could be countoruing an image.

\end{document}

